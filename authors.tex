\author{Nick Merrill, Max T Curran, John Chuang}
\affiliation{%
  \institution{BioSENSE, UC Berkeley School of Information}
  \city{Berkeley} 
  \state{California, USA} 
}
\email{ffff, mtcurran, john.chuang @berkeley.edu}

\hypersetup{
 pdfauthor={},
 pdftitle={},
 pdfkeywords={},
 pdfsubject={},
 pdfcreator={Emacs 25.1.1 (Org mode 9.0.4)}, 
 pdflang={English}}

\begin{abstract}
\textit{Passthoughts authentication}, in which a user thinks a secret thought to log in to services or devices,
provides two factors of authentication (knowledge and inherence) in a single step.
Since its proposal in 2005, passthoughts enjoyed a number of successful empirical studies.
In this paper, we renew the promise of passthoughts authentication,
reviewing the main challenges that passthoughts must overcome in order to move from the lab to the real world.
We propose two studies,
which seek different angles at the fundamental questions we pose.
We conclude with future oppportunities, and challenges, for passthoughts' broader deployment as a tool for authentication.
In doing so, we raise novel possibilities for authentication broadly,
such as ``organic passwords'' that change naturally over time,
or systems that reject users who are not acting quite ``like themselves.''
\end{abstract}

\keywords{passthoughts, authentication, usable security}

