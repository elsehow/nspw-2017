\author{Nick Merrill, Max Curran, John Chuang}
\affiliation{%
  \institution{BioSENSE, UC Berkeley School of Information}
  \city{Berkeley} 
  \state{California, USA} 
}
\email{ffff, mtcurran, john.chuang @berkeley.edu}

\hypersetup{
 pdfauthor={},
 pdftitle={},
 pdfkeywords={},
 pdfsubject={},
 pdfcreator={Emacs 25.1.1 (Org mode 9.0.4)}, 
 pdflang={English}}

\begin{abstract}
\textit{Passthoughts authentication}, in which a user thinks a secret thought to log in to services or devices,
allows for two factors of authentication (knowledge and inherence) in a single step.
Since it was first proposed in 2005, passthoughts has seen a number of successful empirical studies.
In this paper, we renew and justify the importance and promise of passthoughts authentication.
At the same time, we review the main challenges passthoughts must overcome in order to move from the lab to the real world.
We propose two studies that could make significant progress on this front,
both of which seek different angles around the fundamental questions we pose.
We conclude with future oppportunities, and challenges, for passthoughts' broader deployment as a tool for authentication.
\end{abstract}

\keywords{passthoughts, authentication, usable security}

