\author{Nick Merrill, Max Curran, John Chuang}
\affiliation{%
  \institution{BioSENSE, UC Berkeley School of Information}
  \city{Berkeley} 
  \state{California, USA} 
}
\email{ffff@berkeley.edu}

\hypersetup{
 pdfauthor={},
 pdftitle={},
 pdfkeywords={},
 pdfsubject={},
 pdfcreator={Emacs 25.1.1 (Org mode 9.0.4)}, 
 pdflang={English}}

\begin{abstract}
While brain-computer interfaces are used by some individuals with disabilities,
passthought authentication stands a chance at becoming the first brain-computer interface to reach wide, consumer adoption.
However, to move passthoughts out of the lab and into the world, 
we will need both quantitative data about EEG signals 
and rich, qualitative data about user beliefs surrounding EEG specifically, and the possibility of mind-reading devices generally.
This paper proposes two studies, both of which seek different angles around these fundamental questions.
We conclude with future oppportunities, and challenges, for the broader passthoughts acceptance of passthoughts as a tool for authentication.
\end{abstract}

\keywords{passthoughts, authentication, usable security}

